\section{Methodology}
Xxx

\subsection{Experiments}
\subsubsection{Studies}
Some are choice experiments, some expect a number and then compare the answers between different questioning types.

\subsubsection{Scenarios}
- Normal (replication of original study) \\
- Random values \\
- Explicitly prompt to behave humanlike \\

Also describe how the normal prompt is structured.


\subsection{Bias selections}
\par Former research revolving around cognitive biases has shown that there are numerous biases influencing human behavior. Thus, we have to sample a concise yet ideally comprehensive sample of biases to get more generalizable analysis results. To narrow down the range of biases, we focus on biases affecting economic decision-making processes. In particular, we recreate and perhaps slightly modify experiments to make sure they contain an economic decision-making component. 

\setlength{\parindent}{20pt}
\par \textbf{Anchoring bias} The bias known as anchoring is often used to explain why people tend to anchor their estimates around a given value. The central tendency bias is particularly relevant in economic decision-making as it can lead to suboptimal decisions. We recreate the experiment from \cite{tversky1974judgment} where participants are first asked whether the portion of African countries in the United Nations is higher or lower than a certain number and afterwards estimate the exact percentage.

\par \textbf{Category size bias} Decision-making is often influenced by the way alternatives are presented. The category size bias in particular focuses on whether alternatives are presented in a categorized manner and how the resulting categories are distributed (\cite{isaac2014judging}). For example, an investor's expectation about the performance of a particular stock could be influenced by the number of other stocks in the portfolio that belong to the same industry. Similarly, \cite{tversky1994support} showed that participants judged the probability of dying of unnatural cases different when the other causes were presented as one category (natural causes) or individually. To test for the category size bias, we use an experiment from \cite{isaac2014judging} where participants should estimate the probability of randomly selecting a ball from a lottery, itself containing balls with three different colors and varying category sizes.

\par \textbf{Central tendency bias} Humans generally tend towards the mean of a scale when asked to estimate a value.

\par \textbf{Endowment effect} The human tendency to value objects of their endowment higher than if they did not own them is known as the endowment effect. Further, they demand more when giving up the item compared to acquiring it. This effect is often explained as a byproduct of loss aversion (\cite{kahneman1990experimental}). The effect is independent of whether sellers actually earn money or exchange similarly valued goods (\cite{knetsch1989endowment}), though recent research such as \cite{weaver2012reference} suggest that other effects (e.g. fear of financial disadvantage) could be causes. The classic example to illustrate the endowment effect is comparing the willingness to pay for a mug versus the willingness to accept compensation for a mug. The results show that participants owning the mug valued it at more than double the value than the other participants (\cite{kahneman1990experimental}).

\par \textbf{Gambler's fallacy} This bias refers to the human tendency to believe that the probability of a certain event occurring is influenced by the frequency of past events, despite each event being independent (\cite{bar1991perception,kovic2019gambler}). Known as an "insensitivity to sample size", humans also often tailor their decisions to a small sample size which, in their mind, represents the distribution of the larger sample. With regard to gambler's fallacy, humans quickly adapt their judgement based on the law of large numbers even though the law of small numbers is present(\cite{tversky1974judgment}). The shifts in probability perceptions are thus major causes for biased (economic) decision-making. While there has been research on the mechanisms influencing the fallacy (for example the presentation of information (\cite{barron2010role})) as well as possible side effects and fallacies (\cite{kovic2019gambler}), we focus on assessing whether the models are prone to the fallacy with solely head coin flips or balanced coin flips (similar to the Monte Carlo fallacy).

\par \textbf{Illusory correlation} This bias hinges on the idea of gambler's fallacy that is that humans tend to see associations between independent, random events.

\par \textbf{Incentivization} 

\par \textbf{Loss aversion} Loss aversion describes the human habit to prefer avoiding losses over acquiring gains of the same value  (\cite{liu2023review}). Within their research on prospect theory (decision-making biases under risk and uncertainty), \cite{tversky1992advances} estimated that losses are twice as impactful as gains. Closely related to the framing effect, the presentation of the same base information in more risk-averse and risk-seeking scenarios can have significant impact on human decision-making (\cite{druckman2001evaluating}). This bias is particularly relevant in the context of economic decision-making and has been applied to various fields such as retail sale strategies (discount for additional spending), financial investments (sell winners, hold losers) and more (\cite{liu2023review}). We recreate the experiment from \cite{thaler2015misbehaving} which phrases two scenarios as a financial loss and a gain to test for loss aversion.

\par \textbf{Overjustification bias} 

\par \textbf{Sunk cost fallacy} This effect refers to the human phenomenon to preferring an option due to a prior investment into it (sunk costs) even though a better alternative would be available (\cite{arkes1985psychology}). Due to this, even temporally distant investment decisions can have a substantial impact on the decision-making process. The sunk cost fallacy has also been linked as a side effect to some other cognitive biases, most notably loss aversion or commitment bias (\cite{jarmolowicz2016sunk}). We choose to examine the experiment introduced by \cite{arkes1985psychology} where participants are asked to decide between two ski trip scenarios with different sunk costs.

\par \textbf{Take-the-best heuristic} ? Heuristics not biases (Take-the-last, minimalist)

\par \textbf{Transaction utility theory} 

\par \textbf{Ultimate game} ?


\subsection{Model selections}
\par The development and publication of new models and model architectures is ongoing and rapid. 

Ollama for models, larger models ran on cluster


\subsection{Response analysis}
Somewhere describe what the expected output of the models should look like and what I do if it is different.

\subsubsection{Replicability analysis}
Original studies and compare results. Perhaps a "bias detected" number. If always 100 experiment runs, bias detected is the percentage of runs where the model acted biased. (Between 0 and 1, perhaps normalize)

\subsubsection{Average treatment effects}
Average treatment effects of treated on randomized values as well as explicitly prompting humanlike behavior. The control group is the normal prompt.

\subsubsection{Model metadata analysis}
Are there any trends between newer/larger models?

\subsubsection{Model parameter analysis}
Temperature?