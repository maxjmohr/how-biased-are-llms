\section{Results}
\subsection{Analysis of detected biases}
Xxx

\begin{figure}[htbp]
    \centering
    \includesvg{/Users/mAx/Documents/Master/04/Master_Thesis/02_Thesis/Chapters/04_Results/Overview/heatmap_detections.svg}
    \caption[Heatmap of bias detections grouped by biases and models]{\centering \textit{Detected biases grouped by biases and models. Aggregated values are capped between 0 and 1. Detailed calculation and aggregation of the target variable are described in chapters \ref{methodologies:biasdetector} and \ref{methodologies:analysisbiasmodels}.}}
    \label{fig:detections-heatmap}
\end{figure}

\par We also tested the detected biases for homogeneity. Again grouping the detections by biases and models, we find a strong homogeneity in the effect size variances especially for cases with no bias detections (see figure \ref{fig:homogeneity-heatmap}). Applying the 75\% rule by Hunter and Schmidt (REF), we can particularly assume homogeneity for non-bias detections. One exception is the anchoring bias in phi3:medium, where we find a high homogeneous bias detection. Phi3:medium generally stands out with a high homogeneity across its effect sizes (except for the framing effect), indicating a high reliability of the effect size estimates and in turn the model responses. 

\begin{figure}[htbp]
    \centering
    \includesvg{/Users/mAx/Documents/Master/04/Master_Thesis/02_Thesis/Chapters/04_Results/Homogeneity/homogeneity_heatmap.svg}
    \caption[Heatmap of homogeneity grouped by biases and models]{\centering \textit{Homogeneity of capped effect sizes (between 0 and 1) grouped by biases and models. Homogeneity is calculated as the ratio of variance due to sample size and observed variance (detailed in chapter \ref{methodologies:analysisbiasmodels}).}}
    \label{fig:homogeneity-heatmap}
\end{figure}

\subsection{Scenario impact}
Xxx

\subsection{Model analysis}
Xxx

\subsection{Interactions}
Xxx
