\section{Theoretical background}

\subsection{Past studies on human behavioral effects}
Humans are constantly exposed to decision making. Decisions can vary between very simple and complex ones. In studying the decision processes of humans, researchers started seeing the human species as a rational species that makes decisions based on logic and reasoning (\cite{juarez2018analyzing}). However, gaps in these theories such as missing information access were identified quickly. This led to the development of the bounded rationality theory by Herbert Simon {simon1955behavioral}. The theory suggests that humans are not always rational and that they make decisions based on the information available to them. This theory was further developed by Daniel Kahneman and Amos Tversky, who introduced the concept of cognitive biases {kahneman1974judgment}. Cognitive biases are systematic errors in thinking that affect the decisions and judgments that people make.
It has been estimated that 70\% of all decisions by humans are affected by cognitive biases (\cite{juarez2018analyzing}).

\subsection{Leveraging large language models to simulate human behavior}
A) Recent developments in large language model\\
B) The exposure of human behavioral patterns in the models\\
How could models pick up biases? A) Data (texts of humans e.g.), B) Training and Learning (RLHF)

\subsection{Meta analysis techniques}